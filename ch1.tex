% !TEX Root = ./HTNotes-Demo.tex
% !TEX program = xelatex
\section{title} \label{sec:title}

\subsection{title} \label{sbsec:title}

\begin{Win}{displayTitle}
  天气好吗?吃饱了吗?
\end{Win}

\begin{WinOutput}{
  outputContext
}{displayTitle}
  天气好吗?吃饱了吗?
\end{WinOutput}

\begin{Mac}{displayTitle}
  天气好吗?吃饱了吗?
\end{Mac}

\begin{MacOutput}{
  outputContext
}{displayTitle}
  天气好吗?吃饱了吗?
\end{MacOutput}

\begin{GitOutput}[bash]{
    outputContext
  }{displayTitle}
  #include<stdio.h>
  #include<string.h>
  int main() {
    printf("hello, world");
    return 0;
  }
\end{GitOutput}

\begin{Git}[c]{displayTitle}
  #include<stdio.h>
  #include<string.h>
  int main() {
    printf("hello, world");
    return 0;
  }
\end{Git}

\begin{Code}[c]{displayTitle}
#include<stdio.h>
#include<string.h>
int main() {
  printf("hello, world");
  return 0;
}
\end{Code}

  \begin{Code}[matlab][m:2][MATLAB]{Matlab求解}
    b1 = [-6;0];
    a1 = [-3 -2;1 0];
    d1 = [0;5/2];
    format rat;
    alpha1 = b1\(a1*d1) % 右乘a1
  \end{Code}

\WinFile[Python]{testInput}{helloworld.py}
\MacFile[Python]{testInput}{helloworld.py}
\GitFile[Python]{testInput}{helloworld.py}
\CodeFile{testInput}{helloworld.py}

\begin{minted}{HTML}
<html>
<head>
<!-- 网页头部, 用以申明网页标题和引入的格式 -->
<title>你好, 世界!</title>
</head>
<body>
<!-- 网页主体 -->
</body>
</html>
\end{minted}

\begin{Git}{TensorFlow@xubuntu:$\sim$}
TensorFlow@xubuntu:~$
---------- TRAINING -------------
Epoch: 004/020 cost: 1.41867 train_accuray:0.63000 test_accuray:0.62820
Epoch: 008/020 cost: 0.98922 train_accuray:0.71000 test_accuray:0.72040
Epoch: 012/020 cost: 0.81536 train_accuray:0.70000 test_accuray:0.76270
Epoch: 016/020 cost: 0.71605 train_accuray:0.83000 test_accuray:0.78570
Epoch: 020/020 cost: 0.65017 train_accuray:0.82000 test_accuray:0.80130
\end{Git}

\begin{GitOutput}{Output Data}{TensorFlow@xubuntu:$\sim$}
TensorFlow@xubuntu:~$ python3 test.py
Extracting data/train-images-idx3-ubyte.gz
Extracting data/train-labels-idx1-ubyte.gz
Extracting data/t10k-images-idx3-ubyte.gz
Extracting data/t10k-labels-idx1-ubyte.gz
train shape: (55000, 784) (55000, 10)
test  shape: (10000, 784) (10000, 10)
----------MNIST loaded----------------
NeuralNetwork Ready!
2018-08-02 00:11:15.818190: I tensorflow/core/platform/cpu_feature_guard.cc:137] Your CPU supports instructions that this TensorFlow binary was not compiled to use: SSE4.1 SSE4.2 AVX AVX2 FMA
2018-08-02 00:11:16.044897: I tensorflow/core/common_runtime/gpu/gpu_device.cc:1030] Found device 0 with properties:
name: GeForce GTX 1080 Ti major: 6 minor: 1 memoryClockRate(GHz): 1.582
pciBusID: 0000:0a:00.0
totalMemory: 10.92GiB freeMemory: 3.04GiB
2018-08-02 00:11:16.044948: I tensorflow/core/common_runtime/gpu/gpu_device.cc:1120] Creating TensorFlow device (/device:GPU:0) -> (device: 0, name: GeForce GTX 1080 Ti, pci bus id: 0000:0a:00.0, compute capability: 6.1)
---------- TRAINING -------------
Epoch: 004/020 cost: 1.41867 train_accuray:0.63000 test_accuray:0.62820
Epoch: 008/020 cost: 0.98922 train_accuray:0.71000 test_accuray:0.72040
Epoch: 012/020 cost: 0.81536 train_accuray:0.70000 test_accuray:0.76270
Epoch: 016/020 cost: 0.71605 train_accuray:0.83000 test_accuray:0.78570
Epoch: 020/020 cost: 0.65017 train_accuray:0.82000 test_accuray:0.80130

Extracting data/train-images-idx3-ubyte.gz
Extracting data/train-labels-idx1-ubyte.gz
Extracting data/t10k-images-idx3-ubyte.gz
Extracting data/t10k-labels-idx1-ubyte.gz
train shape: (55000, 784) (55000, 10)
test  shape: (10000, 784) (10000, 10)
----------MNIST loaded----------------
NeuralNetwork Ready!
2018-08-02 00:11:15.818190: I tensorflow/core/platform/cpu_feature_guard.cc:137] Your CPU supports instructions that this TensorFlow binary was not compiled to use: SSE4.1 SSE4.2 AVX AVX2 FMA
2018-08-02 00:11:16.044897: I tensorflow/core/common_runtime/gpu/gpu_device.cc:1030] Found device 0 with properties:
name: GeForce GTX 1080 Ti major: 6 minor: 1 memoryClockRate(GHz): 1.582
pciBusID: 0000:0a:00.0
totalMemory: 10.92GiB freeMemory: 3.04GiB
2018-08-02 00:11:16.044948: I tensorflow/core/common_runtime/gpu/gpu_device.cc:1120] Creating TensorFlow device (/device:GPU:0) -> (device: 0, name: GeForce GTX 1080 Ti, pci bus id: 0000:0a:00.0, compute capability: 6.1)
\end{GitOutput}