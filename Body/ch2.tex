% !TEX Root = ../HTNotes-Demo.tex
% !TEX program = xelatex
\section{终端盒子使用} \label{sec:终端盒子使用}
  \subsection{win10 终端} \label{sbsec:win10 终端}
    注意: win10 盒子 minted 语法高亮的语言默认为 shell(sh).

    \mybox{不带输出的盒子}
      \begin{Code}[latex][][LaTeX]{win10 不带输出的盒子的使用语法}
        \begin{Win}[minted高亮的语言]{盒子标题}
          待显示内容
        \end{Win}
      \end{Code}
      效果如下:
      \begin{Win}{win10不带输出的盒子的使用}
        待显示内容
      \end{Win}

    \mybox{带输出的盒子}
      \begin{Code}[LaTeX]{win10 带输出的盒子的使用语法}
        \begin{WinOutput}[minted高亮的语言]{盒子标题}{
          待输出的内容
        }
          待显示内容
        \end{WinOutput}
      \end{Code}
      效果如下:
      \begin{WinOutput}{win10 带输出的盒子的使用语法}{
        待输出的内容
      }
        待显示内容
      \end{WinOutput}

    \mybox{导入文件使用}
      \begin{Code}[LaTeX]{导入文件使用盒子使用语法}
        \WinFile[Python]{导入文件使用盒子}{导入文件的路径, 须加上后缀名}
      \end{Code}
      效果如下:
      \WinFile[Python]{导入文件使用盒子}{Codes/helloworld.py}

  \subsection{Mac 终端} \label{sbsec:Mac 终端}
    注意: Mac 盒子 minted 语法高亮的语言默认为 shell(sh).

    \mybox{不带输出的盒子}
      \begin{Code}[latex][][LaTeX]{Mac 不带输出的盒子的使用语法}
        \begin{Mac}[minted高亮的语言]{盒子标题}
          待显示内容
        \end{Mac}
      \end{Code}
      效果如下:
      \begin{Mac}{Mac不带输出的盒子的使用}
        待显示内容
      \end{Mac}

    \mybox{带输出的盒子}
      \begin{Code}[LaTeX]{Mac 带输出的盒子的使用语法}
        \begin{MacOutput}[minted高亮的语言]{盒子标题}{
          待输出的内容
        }
          待显示内容
        \end{MacOutput}
      \end{Code}
      效果如下:
      \begin{MacOutput}{Mac 带输出的盒子的使用语法}{
        待输出的内容
      }
        待显示内容
      \end{MacOutput}

    \mybox{导入文件使用}
      \begin{Code}[LaTeX]{导入文件使用盒子使用语法}
        \MacFile[Python]{导入文件使用盒子}{导入文件的路径, 须加上后缀名}
      \end{Code}
      效果如下:
      \MacFile[Python]{导入文件使用盒子}{Codes/helloworld.py}

  \subsection{Ubantu 终端} \label{sbsec:Ubantu 终端}
    注意: Ubantu 盒子 minted 语法高亮的语言默认为 bash.

    \mybox{不带输出的盒子}
      \begin{Code}[latex][][LaTeX]{Ubantu 不带输出的盒子的使用语法}
        \begin{Git}[minted高亮的语言]{盒子标题}
          待显示内容
        \end{Git}
      \end{Code}
      效果如下:
      \begin{Git}{Ubantu不带输出的盒子的使用}
        待显示内容
      \end{Git}

    \mybox{带输出的盒子}
      \begin{Code}[LaTeX]{Ubantu 带输出的盒子的使用语法}
        \begin{GitOutput}[minted高亮的语言]{盒子标题}{
          待输出的内容
        }
          待显示内容
        \end{GitOutput}
      \end{Code}
      效果如下:
      \begin{GitOutput}{Ubantu 带输出的盒子的使用语法}{
        待输出的内容
      }
        待显示内容
      \end{GitOutput}

    \mybox{导入文件使用}
      \begin{Code}[LaTeX]{导入文件使用盒子使用语法}
        \GitFile[Python]{导入文件使用盒子}{导入文件的路径, 须加上后缀名}
      \end{Code}
      效果如下:
      \GitFile[Python]{导入文件使用盒子}{Codes/helloworld.py}

\section{代码环境使用} \label{sec:代码环境使用}
  \subsection{直接输入代码环境} \label{sbsec:直接输入代码环境}
    \begin{Win}[LaTeX]{直接输入代码环境使用语法}
      \begin{Code}[minted高亮语言][交叉引用标签][右上角显示语言]{显示的标题}
        代码
      \end{Code}
    \end{Win}
    效果如下:
    \begin{Code}[sh][交叉引用标签][右上角显示语言]{显示的标题}
      代码
    \end{Code}

  \subsection{外部文件导入代码环境} \label{sbsec:外部文件导入代码环境}
    \begin{Win}[LaTeX]{直接输入代码环境使用语法}
      \CodeFile[minted高亮语言][交叉引用标签][右上角显示语言]{显示的标题}{导入文件的路径, 须加上后缀名}
    \end{Win}
    效果如下:
    \CodeFile{显示的标题}{Codes/helloworld.py}
