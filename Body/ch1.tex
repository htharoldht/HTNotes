% !TEX root = ../HTNotes-Demo.tex
% !TEX program = xelatex

本文档类已加载的宏包如下:

\begin{tabular*}{\textwidth}{cccccccc}
\toprule
  xkyval & etoolbox & xcolor & graphicx & ctex & geometry & fancyhdr & pifont \\
  calc & multicol & amsmath & amssymb & bm & extarrows & ntheorem & caption \\
  subcaption & varwidth & enumitem & footmisc & ean13isbn & fancybox & tcolorbox & titletoc \\
  tabu & booktabs & CJKnumb & ocgx2 & hyperref & mdframed & printlen & \\
\bottomrule
\end{tabular*}

\section{模板默认设置} \label{sec:模板默认设置}
\begin{Code}[LaTeX]{模板最小运行示例默认设置}
  % !TEX program = xelatex %% 使用 xelatex 编译
  % !TEX encode = UTF-8 %% 文件编码采用 UTF-8
  \documentclass{HTNotes}

  \begin{document}
  \maketitle % 添加封面
  \makeflypage

  \frontmatter
  \tableofcontents

  \mainmatter
  \chapter{foo}
    \section{foo}
    ...

  \backmatter
  \makebackcover
  \end{document}
\end{Code}
\begin{Win}{几点说明}
  1. setting.tex需要放在与HTNotes,cls的同一目录下, 为自动加载.
  2. settings.tex可以作为导言区使用, 以避免导言区过长, 影响可读性.
\end{Win}

\section{模板参数说明} \label{sec:模板参数说明}
  参数均采用键值对的方式指定, 接口如下
  \begin{Code}[LaTeX]{接口}
  \HTset{
    key = val,
    key = val,
    ...
  }
  \end{Code}

  \subsection{封面相关元素设置} \label{sbsec:封面相关元素设置}
    \begin{enumerate}
      \item 颜色 \verb|cvcolor|

        模板提供了 \verb|orange|, \verb|green|, \verb|violet|, \verb|blue|四种颜色.

        默认为 \verb|blue|.

      \item {丛书图标} \verb|logo|

        模板提供了 \verb|Lion|, \verb|Mine|, \verb|user1|, \verb|user2四种图标|.

        用户只需要将图片命名为相应的选项即可.

        默认为 \verb|Lion|.

      \item 丛书名称 \verb|Series|

        可为空. 默认为计算机应用技术丛书.

      \item 副标题 \verb|SubTitle|

        可为空. 默认为这里可以放一个副标题.

      \item 封面书籍简介 \verb|Introduction|

        为文字或图片. 默认为一大段话.

      \item 版次 \verb|Edition|

        可为空. 默认为1.

      \item 出版社 \verb|Publisher|

        默认为 \LaTeX Studio 出版社.
    \end{enumerate}

  \subsection{PDF信息设置} \label{sbsec:PDF信息设置}
    \begin{enumerate}
      \item PDF主题 \verb|Subject|

        默认为空.

      \item PDF 创建程序 \verb|Producer|

        默认为 XeLaTeX+TeXLive 2018.

      \item PDF 关键词 \verb|Keywords|

        默认为空.
    \end{enumerate}

  \section{版权页信息设置} \label{sec:版权页信息设置}
    \begin{enumerate}
      \item 字数统计 \verb|Words|

        默认为666千字.

      \item 反馈邮箱 \verb|Email|

        默认为 \href{mailto:htharoldht@Gmail.com}{我的邮箱}.

      \item 反馈网站 \verb|Website|

        默认为 \href{https://htharoldht.com}{我的网站}
    \end{enumerate}

  \subsection{空白页信息设置} \label{sbsec:空白页信息设置}
    在扉页和前言也之间有时会需要一页作为编委会介绍.

    空白页 \verb|ExtraPage|

    只需要填写内容即可, 有内容即为真, 无则隐藏.

    如 \verb|ExtraPage={foo}|

    \begin{enumerate}
      \item 前言内容 \verb|Preface|

        直接填写内容即可, 默认为中文乱数假文.

      \item 前言作者 \verb|Writer|

        默认为书籍作者.
    \end{enumerate}

  \subsection{封底信息设置} \label{sbsec:封底信息设置}
    \begin{enumerate}
      \item 封底后记 \verb|Summary|

        可为空. 默认为中文乱数假文.

      \item 编辑 \verb|Editor|

        默认为张三.

      \item 封面设计 \verb|Designer|

        默认为李四.

      \item 价格 \verb|Price|

        默认为806, 我的大学宿舍.

      \item ISBN 编号. \verb|ISBN|

        默认为978-7-302-11622-6.
    \end{enumerate}

  \subsection{其他几点说明} \label{sbsec:其他几点说明}
    \begin{enumerate}
      \item 若要使用用户字体, 须用
        \mint{LaTeX}|\documentclass[customfont=true]{HTNotes}|
        且需要安装有思源宋体、思源黑体、文泉驿微米黑、造字工坊的刻宋、郎宋、黄金时代

      \item 答案初始显示 \verb|answer|

        可为 \verb|true|或\verb|false|

        默认为false,即初始不显示答案.

      \item 若使用了minted, 则编译选项须有--shell-escape
    \end{enumerate}